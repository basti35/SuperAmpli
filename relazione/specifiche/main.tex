%       dFile: main.tex
%     Created: mer giu 15 06:00  2011 C
% Last Change: mer giu 15 06:00  2011 C
%
\chapter{Specifiche di progetto}
%\begin{flushright}
%  \textit{Andrew: ``Uno \`e lieto di poter servire''\\L'uomo bicentenario (1999)}
%\end{flushright}
\vspace {30pt}

Riepilogo delle specifiche imposte per l'amplificatore e di quelle imposte per
la progettazione.
\newpage

\section{Specifiche imposte}
  Si progetti un amplificatore audio con le seguenti caratteristiche:
  \begin{description}
    \item[Impedenza di carico]: 8 \ohm (il carico dell'amplificatore sar\`a un
      tweeter);
    \item[Banda]: [3,5; 21] kHz;
    \item[Tipo di alimentazione]: singola;
    \item[Distorsione massima in potenza]: 0,1\%;
    \item[Massima potenza in uscita richiesta]: 50W;
    \item[Massima potenza in ingresso]: 1mW;
    \item[Max ripple del guadagno in banda]: 0.5dB;
    \item[Tolleranze per componenti passivi]: [0,5\%; 1\%];
    \item[Tolleranze per componenti attivi]: 50\%;
    \item[Resistenza interna del generatore di segnale]: 50\ohm;
  \end{description}
  \subsection{Restrizioni ulteriori}
  Nello sviluppo del sistema, tenere conto dei seguenti vincoli di progetto:
  \begin{itemize}
    \item per realizzare l'amplificatore occorre usare componenti commerciali e
      come dispositivi attivi solamente transistor MOSFET:
    \item il numero di componenti deve essere minimo
    \item il consumo di potenza deve essere minimo;
  \end{itemize}
  \subsection{Conclusione del progetto}
  Affinch\`e il progetto possa essere considerato concluso:
  \begin{enumerate} 
    \item l'amplificatore deve essere stabile;
    \item lo schematico del circuito dovr\`a soddisfare le specifiche di
      progetto.
  \end{enumerate}
\section{Consigli di progettazione}
  Durante la progettazione dell'amplificatore tenere conto della seguente
  traccia di massima:
  \begin{enumerate}
    \item Scegliere l'arhitettura del sistema:
      \begin{itemize}
	\item Retroazione?
	\item Di che tipo?
	\item Perch\`e?
      \end{itemize}
    \item Definire le specifiche di massima per ogni singolo blocco che
      comporr\`a il sistema.
    \item Partendo dall'ultimo, progettare ogni blocco, includendo gli effetti
      di carico degli stadi successivi e il comportamento a largo segnale.
      Verificare il comportamento complessivo dell'amplificatore.
    \item Unire i vari stadi e verificare il comportamento complessivo
      dell'amplificatore.
    \item Verificare la stabilit\`a ed eventualmente compensare l'amplificatore.
      Nel caso l'amplificatore sia accoppiato in AC, verificare anche la
      stabilit\`a a bassa frequenza.
    \item Verificare tutte le specifiche dell'amplificatore completo.
  \end{enumerate}


%  \begin{figure}[htbp]
%    \center
%    \includegraphics[height=12cm]{./001/img/schema_robot_mob}
%    \caption{Schema di principio del funzionamento di un robot mobile}
%    \label{fig:schema_rob_mob}
%  \end{figure}

\clearpage
